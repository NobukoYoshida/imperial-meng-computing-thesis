\section{The Scribble Framework}
\label{section:scribble}

Development begins by specifying the permitted communications between
participants as a \textit{global protocol} in Scribble \cite{Scribble}, a
MPST-based protocol specification language and code generation toolchain.
We specify the \textit{Noughts and Crosses} game as a Scribble protocol in
\cref{lst:game}.

\begin{figure}[!h]
\begin{lstlisting}[
	language=Scribble
]
module NoughtsAndCrosses;
type <typescript> "Coordinate" from "./Types" as Point;	// Position on board

global protocol Game(role Svr, role P1, role P2) {
  Pos(Point) from P1 to Svr;
  choice at Svr {
    Lose(Point) from Svr to P2; Win(Point) from Svr to P1;
  } or {
    Draw(Point) from Svr to P2; Draw(Point) from Svr to P1;
  } or {
    Update(Point) from Svr to P2; Update(Point) from Svr to P1;
    do Game(Svr, P2, P1); // Continue the game with player roles swapped
  }
}
\end{lstlisting}
\captionof{lstlisting}{Main body of the \textit{Noughts and Crosses} protocol.}
\label{lst:game}
\end{figure}

We leverage the Scribble toolchain to check for protocol
well-formedness.
This directly corresponds to multiparty session
type theory \cite{MPST}:
\fz{Cite the featherweight scribble paper for the correspondence?}
a Scribble protocol maps to some \textit{global type}, and protocol
verification implements the algorithmic projection defined in \cite{MPST} to
derive valid local type \textit{projections} for all participants.
% Note: Scribble Projection is not MPST projection, because you can have
% protocols with choices that send to different roles.
We obtain a set of \textit{endpoint protocols} -- one for each role from a
well-formed global protocol.
An endpoint protocol only preserves the interactions defined by the global
protocol in which the target role is involved, and corresponds to an equivalent
\textit{Endpoint Finite State Machine} (EFSM) \cite{ICALP13CFSM}.
We use the EFSMs as a basis for API generation and adopt the formalisms in
\cite{Hybrid2016}.


