\section{Related Works}
Native language support for first-class linear channel resources \cite{ATS} and
code generation are two main approaches for incorporating MPST into application
development.
The latter closely relates to our proposal;
we highlight two areas of existing works under this approach that motivate our
design choice.

\paragraph{Endpoint API generation}
Neykova and Yoshida targeted Python applications and the generation of runtime
monitors \cite{Python2017} to dynamically verify communication patterns.
Whilst the same approach could be applied to JavaScript, we can provide more
static guarantees with TypeScript's gradual typing system and compiler. Scribble-Java \cite{Hybrid2016} proposed to encode the EFSM
states and transitions as classes and instance methods respectively, with
behavioural typing achieved statically by the type system and channel linearity
guarantees achieved dynamically since channels are exposed and
aliasing is not monitored.

\paragraph{Session types in web development}
King et al. \cite{PureScript2019} targeted web development in PureScript using the
\textit{Concur UI} framework and proposed a type-level encoding of EFSMs as
multi-parameter type classes.
However, it presents a trade-off between achieving static linearity guarantees
from the type-level EFSM encoding under the expressive type system and
providing an intuitive development experience to developers, especially given
the prevalence of JavaScript and TypeScript applications in industry. Fowler \cite{MVU2019} focused on applying binary session types in front-end web
development and presented approaches that tackle the challenge of guaranteeing
linearity in the event-driven environment.

Our work applies the aforementioned approaches in a \textit{multiparty} context
using industrial tools and practices to ultimately encourage MPST-safe web
application development workflows in industry.

%\paragraph{Typestate programming} \dots

